\documentclass[9pt]{article}

\usepackage{fullpage}
\usepackage{hyperref}
\usepackage{enumitem}
\usepackage{multicol}
\usepackage[normalem]{ulem}

\addtolength{\topmargin}{-.25in}
\addtolength{\textheight}{0.5in}
\setlength{\parindent}{0pt}
\setlength{\multicolsep}{2pt}

\title{ASEN 6519-001: Advanced Survey of Sequential Decision Making}
\author{Zachary Sunberg}
\date{Fall 2021}

\begin{document}

\maketitle

\section*{Prerequisites}

ASEN 5519 Decision Making under Uncertainty, research experience in robotics or artificial intelligence, or permission from the instructor.

\section*{Learning Objectives}

\begin{enumerate}[nosep]
    \item Obtain a broad understanding of models and algorithms for sequential decision making
    \item Understand theoretical tools for analyzing sequential decision making
    \item Communicate about recent research in sequential decision making
    \item Become familiar with and investigate open research questions in sequential decision making
\end{enumerate}

\section*{List of Topics}

(See Canvas for detailed schedule.)

\begin{enumerate}[nosep]
    \item Survey of popular reinforcement learning algorithms
    \item Applications of sequential decision-making algorithms
    \item Theoretical tools for characterizing sequential decision-making algorithms
    \item Important properties of select algorithms
    \item Advanced online and offline algorithms for solving POMDPs
    \item Leading edge of reinforcement learning research
    \item Guest lectures on model-learning and reachable set computation
\end{enumerate}

\section*{Learning Technology}

\textbf{Canvas} will be the main hub for the course. A detailed schedule will be posted there.

\textbf{Piazza} will host course discussions. Students are encouraged to ask questions there.

\section*{Assignments and Grading}

\textbf{40\% Presentations.}
Each student will read and prepare presentations for two papers that will be assigned after the first week. For each paper there will be three requirements:
\begin{itemize}[nosep]
    \item Draft presentation: Must be uploaded several days before the date of the presentation - see canvas for details.
    \item In-class presentation and discussion: You will deliver a 20-25 minute presentation summarizing the paper and lead a 10-15 minute discussion.
    \item Two practice exercises to help fellow students prepare for the midterms and form the basis for midterm questions.
\end{itemize}

\textbf{20\% Quizzes.}
There will be two take-home midterm quizzes.

\textbf{35\% Final Project.}
A final project chosen by the student that ideally connects to their research. Deliverables will be a 15 minute presentation and a 4-8 page report. There will Project may be completed in teams of up to 2.

\textbf{5\% Participation.}
Students are expected to attend all lectures except for approved reasons such as travel, illness, or religious observances. You will also be expected to ask questions and participate in discussions during the semester.

\section*{Instructor Contact}

Professor Zachary Sunberg\\
AERO 263 \href{mailto://zachary.sunberg@colorado.edu}{zachary.sunberg@colorado.edu}\\
\textbf{Office Hours}
By request: please create a calendar invitation for a time marked free on my calendar at \url{ https://zachary.sunberg.net/contact#calendar} and indicate whether you would like to meet in my office or provide a videoconference link. \\

\section*{Meetings}

Tuesdays and Thursdays 10:05-11:20 am\\
AERO N250 / Zoom Link on Canvas

\section*{Auditing}

Auditing of the course is allowed, however, it is requested (though not required) that auditors volunteer to present one of the papers.

\section*{University Policies}

\textbf{Please find an accessible online copy of the policies here: \url{https://www.colorado.edu/academicaffairs/student-syllabus-statements}. The linked document contains urls to other department websites that are not included in the text below.}

{\small

    \subsection*{Classroom Behavior}
Both students and faculty are responsible for maintaining an appropriate learning environment in all instructional settings, whether in person, remote or online. Those who fail to adhere to such behavioral standards may be subject to discipline. Professional courtesy and sensitivity are especially important with respect to individuals and topics dealing with race, color, national origin, sex, pregnancy, age, disability, creed, religion, sexual orientation, gender identity, gender expression, veteran status, political affiliation or political philosophy.  For more information, see the policies on classroom behavior and the Student Conduct \& Conflict Resolution policies. 

\subsection*{Requirements for COVID-19}

As a matter of public health and safety due to the pandemic, all members of the CU Boulder community and all visitors to campus must follow university, department and building requirements and all public health orders in place to reduce the risk of spreading infectious disease. Students who fail to adhere to these requirements will be asked to leave class, and students who do not leave class when asked or who refuse to comply with these requirements will be referred to Student Conduct and Conflict Resolution. For more information, see the policy on classroom behavior and the Student Code of Conduct. If you require accommodation because a disability prevents you from fulfilling these safety measures, please follow the steps in the “Accommodation for Disabilities” statement on this syllabus.

As of Aug. 13, 2021, CU Boulder has returned to requiring masks in classrooms and laboratories regardless of vaccination status. This requirement is a temporary precaution during the delta surge to supplement CU Boulder’s COVID-19 vaccine requirement. Exemptions include individuals who cannot medically tolerate a face covering, as well as those who are hearing-impaired or otherwise disabled or who are communicating with someone who is hearing-impaired or otherwise disabled and where the ability to see the mouth is essential to communication. If you qualify for a mask-related accommodation, please follow the steps in the “Accommodation for Disabilities” statement on this syllabus. In addition, vaccinated instructional faculty who are engaged in an indoor instructional activity and are separated by at least 6 feet from the nearest person are exempt from wearing masks if they so choose. 
Students who have tested positive for COVID-19, have symptoms of COVID-19, or have had close contact with someone who has tested positive for or had symptoms of COVID-19 must stay home. 

\subsection*{Accommodation for Disabilities}
If you qualify for accommodations because of a disability, please submit your accommodation letter from Disability Services to your faculty member in a timely manner so that your needs can be addressed.  Disability Services determines accommodations based on documented disabilities in the academic environment.  Information on requesting accommodations is located on the Disability Services website. Contact Disability Services at 303-492-8671 or dsinfo@colorado.edu for further assistance.  If you have a temporary medical condition, see Temporary Medical Conditions on the Disability Services website.

\subsection*{Preferred Student Names and Pronouns}

CU Boulder recognizes that students' legal information doesn't always align with how they identify. Students may update their preferred names and pronouns via the student portal; those preferred names and pronouns are listed on instructors' class rosters. In the absence of such updates, the name that appears on the class roster is the student's legal name.

\subsection*{Honor Code}
All students enrolled in a University of Colorado Boulder course are responsible for knowing and adhering to the Honor Code academic integrity policy. Violations of the Honor Code may include, but are not limited to: plagiarism, cheating, fabrication, lying, bribery, threat, unauthorized access to academic materials, clicker fraud, submitting the same or similar work in more than one course without permission from all course instructors involved, and aiding academic dishonesty. All incidents of academic misconduct will be reported to the Honor Code (honor@colorado.edu); 303-492-5550). Students found responsible for violating the academic integrity policy will be subject to nonacademic sanctions from the Honor Code as well as academic sanctions from the faculty member. Additional information regarding the Honor Code academic integrity policy can be found on the Honor Code website.

\subsection*{Sexual Misconduct, Discrimination, Harassment and/or Related Retaliation}
The University of Colorado Boulder (CU Boulder) is committed to fostering an inclusive and welcoming learning, working, and living environment. CU Boulder will not tolerate acts of sexual misconduct (harassment, exploitation, and assault), intimate partner violence (dating or domestic violence), stalking, or protected-class discrimination or harassment by or against members of our community. Individuals who believe they have been subject to misconduct or retaliatory actions for reporting a concern should contact the Office of Institutional Equity and Compliance (OIEC) at 303-492-2127 or email cureport@colorado.edu. Information about OIEC, university policies, reporting options, and the campus resources can be found on the OIEC website.

Please know that faculty and graduate instructors have a responsibility to inform OIEC when made aware of incidents of sexual misconduct, dating and domestic violence, stalking, discrimination, harassment and/or related retaliation, to ensure that individuals impacted receive information about their rights, support resources, and reporting options.

\subsection*{Religious Holidays}
Campus policy regarding religious observances requires that faculty make every effort to deal reasonably and fairly with all students who, because of religious obligations, have conflicts with scheduled exams, assignments or required attendance.  
See the campus policy regarding religious observances for full details.

}

\end{document}
