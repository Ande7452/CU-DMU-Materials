\documentclass{article}
\usepackage{fullpage,amsmath,amsthm,graphicx,enumitem}
\usepackage{multicol}
\usepackage{booktabs}

\theoremstyle{definition}
\newtheorem{thm}{Theorem}
\newtheorem{question}[thm]{Question}
\newenvironment{solution}{\noindent\textit{Solution:}}{}

\title{ASEN 5519-003 Decision Making under Uncertainty\\
       Homework 1: Probabilistic Models}

\begin{document}

\maketitle

\section{Conceptual Questions}

\begin{question} (20 pts)
    Consider the following joint distribution of three binary-valued random variables, \mbox{$A$, $B$, and $C$}:

    \begin{minipage}{0.23\linewidth}
        \vspace{1em}
    {\tiny
    \begin{tabular}{cccc}
        \toprule
            $A$ & $B$ & $C$ & $P(A,B,C)$ \\
        \midrule
            $0$ & $0$ & $0$ & $0.06$ \\
            $0$ & $0$ & $1$ & $0.15$ \\
            $0$ & $1$ & $0$ & $0.05$ \\
            $0$ & $1$ & $1$ & $0.01$ \\
            $1$ & $0$ & $0$ & $0.14$ \\
            $1$ & $0$ & $1$ & $0.18$ \\
            $1$ & $1$ & $0$ & $0.30$ \\
        \bottomrule
    \end{tabular}
    }
    \end{minipage}
    \begin{minipage}{0.75\linewidth}
        \begin{enumerate}[label=\alph*)]
            \item What is the probability of the outcome $A=1$, $B=1$, $C=1$?
            \item What is the marginal distribution of $A$?
            \item What is the conditional distribution of $A$ given $B=1$ and $C=1$?
        \end{enumerate}
    \end{minipage}
\end{question}

\begin{question} (20 pts)
    Let $B$ be a uniformly-distributed binary random variable and let $A$ be a real-valued random variable with the following conditional distribution:
    $$A \mid B=0 \sim \mathcal{U}(0,2)$$
    $$A \mid B=1 \sim \mathcal{U}(1,3)$$
    \begin{enumerate}[nosep,label=(\alph*)]
        \item Plot or draw the probability density functions for the conditional distribution of $A$.
        \item Plot or draw the marginal density function of $A$.
        \item What is the probability that $A=1.5$?
        \item What is the probability that $A \in [1.5, 1.6]$?
    \end{enumerate}
\end{question}

\begin{question} (20 pts)
    2\% of women at age forty who participate in routine screening have breast cancer. 86\% of those with breast cancer will get positive mammograms. 8\% of those without breast cancer will also get positive mammograms. A woman in this age group had a positive mammogram in a routine screening. What is the probability that she actually has breast cancer?
\end{question}

\begin{question} (20 pts)
Suppose that a stationary stochastic process $\{x_t\}$ is defined by the following equation: $x_{t+1} = 1.5 \, x_t - x_{t-1} + v_{t}$ where $v_t$ are independent, identically distributed random variables with $v_t \sim \mathcal{N}(0.0, 0.2^2)$.
    \begin{enumerate}[nosep,label=(\alph*)]
        \item Simulate and plot 10 20-step trajectories sampled from this process with $x_0 = x_{-1} = 1$ (as always, submit your code for this).
        \item Is this process a Markov process if the state is defined as $x_t$? Why or why not?
        \item If you only had access to the trajectories you plotted what evidence could you use to convince someone that this process is or is not Markov?
        \item What would need to be included in the state at time $t$ to make this a Markov process?
    \end{enumerate}
\end{question}

\section{Programming}

\begin{question} (20 pts)
    Write a rectified linear unit (relu) function in Julia. The relu function is
    \begin{equation}
        f(x) = \begin{cases}
            x \text{ if } x > 0 \\
            0 \text{ otherwise}
        \end{cases}
    \end{equation}
    That is, it is flat to the left of zero and increases linearly to the right. In order to get used to writing efficient Julia code, this function will need to be type stable and return the same type as the argument. That is, if the argument is an \texttt{Int64}, the function should return an \texttt{Int64}; \texttt{UInt8} arguments should result in a \texttt{UInt8}, \texttt{Float64} in \texttt{Float64}, etc., for all numeric types. Moreover, the compiler must be able to infer this (this can be checked with the \texttt{@inferred} macro).

    Evaluate this function with \texttt{DMUStudent.HW1.evaluate} and submit the resulting json file \textit{along with a listing of the code}. A score of 1 will receive full credit.
\end{question}

\end{document}
