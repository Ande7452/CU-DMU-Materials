\documentclass[9pt]{article}

\usepackage{fullpage}
\usepackage{hyperref}
\usepackage{enumitem}
\usepackage{multicol}

\addtolength{\topmargin}{-.25in}
\addtolength{\textheight}{0.5in}
\setlength{\parindent}{0pt}
\setlength{\multicolsep}{2pt}

\title{ASEN 5519-003: Decision Making under Uncertainty}
\author{Zachary Sunberg}
\date{Spring 2021}

\begin{document}

\maketitle

\section*{Prerequisites}

\begin{enumerate}[noitemsep]
    \item Fluency in a high level programming language and willingness to learn Julia.
\end{enumerate}

\section*{Rough Schedule and List of Topics}

(See Canvas for detailed schedule.)

\begin{enumerate}[noitemsep]
    \item \textbf{Probabilistic Models [1/14 - 1/19]}:
        \begin{multicols}{2}
            \begin{itemize}[noitemsep]
                \item Probability
                \item Conditional probability
                \item Markov processes
            \end{itemize}
        \end{multicols}
    \item \textbf{Markov Decision Processes [1/21 - 2/9]}:
        \begin{multicols}{2}
        \begin{itemize}[noitemsep]
            \item Markov decision processes (MDPs)
            \item Value iteration (contraction proof of convergence)
            \item Policy iteration
            \item Approximate dynamic programming
            \item Online tree search
        \end{itemize}
        \end{multicols}
    \item \textbf{Reinforcement Learning [2/11 - 3/4]}:
        \begin{multicols}{2}
        \begin{itemize}[noitemsep]
            \item Exploration and exploitation
            \item Bandits
            \item Model-free RL
            \item Model-based RL
            \item Deep Q learning
            \item Policy gradient
            \item Actor-critic
        \end{itemize}
        \end{multicols}
    \item \textbf{POMDPs [3/9 - 3/20]}:
        \begin{multicols}{2}
        \begin{itemize}[noitemsep]
            \item Hidden Markov models
            \item Bayesian filters
            \item Particle filters
            \item Partially observable Markov decision processes (POMDPs)
            \item Exact POMDP methods
            \item Offline POMDP methods
            \item Online POMDP methods
            \item QMDP
        \end{itemize}
        \end{multicols}
    \item \textbf{Other Topics [4/1-4/29]}: Bayesian Networks, games, alternative optimization objectives (risk averse, robust, constrained), $\rho$-POMDPs, meta-learning, AlphaStar, review of academic publications
\end{enumerate}

\begin{samepage}
\section*{Learning Technology}

\textbf{Canvas} will be the main hub for the course. A detailed schedule and assignments will be posted here.

\textbf{Piazza} will host course discussions. Students are encouraged to ask questions here.
\end{samepage}
\\

\begin{samepage}
\section*{Assignments and Grading}

\textbf{40\% Homework Assignments.}
There will be 6 large homework assignments, due approximately every two weeks. A typical assignment will consist of
\begin{itemize}[nosep]
    \item Several conceptual questions or exercises.
    \item One open-ended problem. You solution will be evaluated locally with obfuscated code and the score submitted to a leaderboard. The best performers will share their solution in class.
\end{itemize}

\textbf{30\% Quizzes.}
There will be three Quizzes consisting of several conceptual questions or exercises. The exact format has not been set, but it will most likely be a timed quiz that can be taken any time during a 24 or 48 hour period.

\textbf{25\% Final Project.}
A final project chosen by the student that ideally connects to their research. Deliverable will be a 4-8 page report. Project may be completed in teams of up to 3.

\textbf{5\% Peer Review.}
You will be assigned 2 project reports from other teams in the class to write peer reviews for.
\end{samepage}

\subsection*{Late Policy}

For \textbf{homework}, there will be a \textbf{10\% penalty} for submitting the assignment late on the day it is due and a \textbf{20\% penalty} for every late day after that. For the \textbf{final project} and \textbf{peer review}, there will be a \textbf{20\% penalty for every late \emph{hour}} (due to the need for quick turnaround). Please use your knowledge of decision making under uncertainty to include appropriate contingency in your plans to avoid these penalties.

\section*{Textbook}

Mykel J. Kochend

\subsection*{Additional References}

\begin{itemize}[noitemsep]
    \item Mykel J. Kochenderfer, Decision Making Under Uncertainty: Theory and Application, MIT Press, 2015. \$70.00, Available online: \url{https://ieeexplore.ieee.org/book/7288640}
    \item Richard S. Sutton and Andrew G. Barto, Reinforcement Learning: An Introduction, 2nd Ed. MIT Press, 2018. \$80.00, Available online: \url{http://incompleteideas.net/book/the-book-2nd.html}
    \item Dimitri P. Bertsekas, Dynamic Programming and Optimal Control, Athena Scientific, 2012 (4th Ed.). \$134.50
\end{itemize}

\vspace{12pt}
\begin{multicols}{2}
    \begin{minipage}{\columnwidth}
\section*{Instructor Contact}

Professor Zachary Sunberg\\
\href{mailto://zachary.sunberg@colorado.edu}{zachary.sunberg@colorado.edu}\\
AERO 263\\
Office Hours:\\
T/TH 11:20 am - 12:20 pm\\
4-5pm the day before any homework is due


\subsection*{Teaching Assistant}

Tucker Farrell\\
\href{mailto://Tucker.Farrell@colorado.edu}{Tucker.Farrell@colorado.edu}\\
Office Hours: W 11am - 12pm
    \end{minipage}

\section*{Meetings}

T/TH 10-11:15 am AERO 114
\end{multicols}

\section*{University Policies}

{\small


\subsection*{Accommodation for Disabilities}
If you qualify for accommodations because of a disability, please submit your accommodation letter from Disability Services to your faculty member in a timely manner so that your needs can be addressed.  Disability Services determines accommodations based on documented disabilities in the academic environment.  Information on requesting accommodations is located on the Disability Services website. Contact Disability Services at 303-492-8671 or dsinfo@colorado.edu for further assistance.  If you have a temporary medical condition or injury, see Temporary Medical Conditions under the Students tab on the Disability Services website.

\subsection*{Classroom Behavior}
Students and faculty each have responsibility for maintaining an appropriate learning environment. Those who fail to adhere to such behavioral standards may be subject to discipline. Professional courtesy and sensitivity are especially important with respect to individuals and topics dealing with race, color, national origin, sex, pregnancy, age, disability, creed, religion, sexual orientation, gender identity, gender expression, veteran status, political affiliation or political philosophy.  For more information, see the policies on classroom behavior and the Student Code of Conduct.
\subsection*{Preferred Student Names and Pronouns}
CU Boulder recognizes that students' legal information doesn't always align with how they identify. Students may update their preferred names and pronouns via the student portal; those preferred names and pronouns are listed on instructors' class rosters. In the absence of such updates, the name that appears on the class roster is the student's legal name.
\subsection*{Honor Code}
All students enrolled in a University of Colorado Boulder course are responsible for knowing and adhering to the Honor Code. Violations of the policy may include: plagiarism, cheating, fabrication, lying, bribery, threat, unauthorized access to academic materials, clicker fraud, submitting the same or similar work in more than one course without permission from all course instructors involved, and aiding academic dishonesty. All incidents of academic misconduct will be reported to the Honor Code (honor@colorado.edu); 303-492-5550). Students found responsible for violating the academic integrity policy will be subject to nonacademic sanctions from the Honor Code as well as academic sanctions from the faculty member. Additional information regarding the Honor Code academic integrity policy can be found at the Honor Code Office website.

\subsection*{Sexual Misconduct, Discrimination, Harassment and/or Related Retaliation}
The University of Colorado Boulder (CU Boulder) is committed to fostering a positive and welcoming learning, working, and living environment. CU Boulder will not tolerate acts of sexual misconduct, intimate partner abuse (including dating or domestic violence), stalking, or protected-class discrimination or harassment by members of our community. Individuals who believe they have been subject to misconduct or retaliatory actions for reporting a concern should contact the Office of Institutional Equity and Compliance (OIEC) at 303-492-2127 or cureport@colorado.edu. Information about the OIEC, university policies, anonymous reporting, and the campus resources can be found on the OIEC website. 
Please know that faculty and instructors have a responsibility to inform OIEC when made aware of incidents of sexual misconduct, discrimination, harassment and/or related retaliation, to ensure that individuals impacted receive information about options for reporting and support resources.
\subsection*{Religious Holidays}
Campus policy regarding religious observances requires that faculty make every effort to deal reasonably and fairly with all students who, because of religious obligations, have conflicts with scheduled exams, assignments or required attendance.  In this class, please contact me at least two weeks in advance if you require religious accommodation.  See the campus policy regarding religious observances for full details.
\\

[accessible online copy of policies: \url{https://www.colorado.edu/academicaffairs/student-syllabus-statements}]
}

\end{document}
