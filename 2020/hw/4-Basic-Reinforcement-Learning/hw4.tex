\documentclass{article}
\usepackage{fullpage,amsmath,amsthm,graphicx,enumitem,amssymb}
\usepackage[hidelinks]{hyperref}
\theoremstyle{definition}
\newtheorem{thm}{Theorem}
\newtheorem{question}[thm]{Question}
\newenvironment{solution}{\noindent\textit{Solution:}}{}

\newcommand{\reals}{\mathbb{R}}

\title{ASEN 6519-007 Decision Making under Uncertainty\\
       Homework 4: Basic Reinforcement Learning}

\begin{document}

\maketitle

\section{Exercises}

\begin{question}
    (20 pts) Using the deep learning library of your choice (e.g. Flux.jl, Knet.jl, Tensorflow, Keras), fit a neural network to approximate the function $f(x) = cos(20\,x^2)$ for the range $x \in [0,1]$. Plot a set of 100 data points fed through the trained model and plot the learning curve.
\end{question}

\begin{question}
    (30 pts) Implement \textbf{two} different traditional or deep learning algorithms to learn a policy for the \texttt{DMUStudent.HW4.gw} grid world environment. Use a discount factor of $\gamma=0.95$ to encourage the agent to reach goals more quickly. Plot a learning curve for each and comment on why one performs better than the other. You must implement these algorithms from scratch yourself; you are not permitted to use libraries specially designed for Reinforcement learning.\footnote{\label{fn:rlinterface}Both \texttt{HW4.gw} and \texttt{HW4.mc} implement the RLInterface.jl interface. Details can be found at \url{https://github.com/JuliaPOMDP/RLInterface.jl}. This interface will allow you to collect data from the environment and evaluate policies with it.}
\end{question}

\section{Challenge Problem}

\begin{question}
    (50 pts) Learn a policy for the mountain car environment \texttt{DMUStudent.HW4.mc}. You may use \emph{any} libraries. A discount factor of $\gamma=0.99$ is used for evaluation. A score of 35 or greater will receive full credit.\footnotemark[1]\footnote{Your submission should be either a \texttt{Function} that takes the state as the single argument and returns an action, or a \texttt{POMDPs.Policy}.}
\end{question}

\end{document}
