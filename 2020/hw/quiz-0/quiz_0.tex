\documentclass{exam}

% \usepackage{amssymb}
\usepackage{pifont}
\usepackage{booktabs}
\usepackage{wrapfig}

\newcommand{\cmark}{\ding{51}}
\newcommand{\xmark}{\ding{55}}
\newcommand{\answer}[0]{\vspace{30pt}}

\addtolength{\topmargin}{-0.5in}
\addtolength{\textheight}{0.5in}

\title{ASEN 6519-007 Quiz 0: Background Assessment}

\begin{document}

\maketitle

    This is a short survey quiz to gauge your background and interests for the course. The quiz is not graded; the purpose is to gauge initial understanding, and we will review the concepts in class if they seem rusty. Don't spend more than a half hour on this. 

\begin{questions}
    \question What other courses related to probability and estimation have you taken? Which other controls or robotics, machine learning, or other related courses? 

    \answer

    \question What programming languages are you familiar with? Rate your familiarity on a scale of 1 (have read about) to 5 (have used daily for years)

    \answer

    \question In general terms, what do you hope to learn from taking this course? Is there any specific topic you would like to see covered in this course?

    \answer

    \question What is a random variable?

    \answer

    \question In your own words, explain what a probability density function is (pdf)? Is it possible for the cumulative density function (cdf) to exist, but not have a finite pdf? Explain.

    \answer

    \question Consider the following joint distribution of three binary-valued random variables, $A$, $B$, and $C$:

    \begin{minipage}{0.23\linewidth}
    {\tiny
    \begin{tabular}{cccc}
        \toprule
            $A$ & $B$ & $C$ & $P(A,B,C)$ \\
        \midrule
            $0$ & $0$ & $0$ & $0.08$ \\
            $0$ & $0$ & $1$ & $0.15$ \\
            $0$ & $1$ & $0$ & $0.05$ \\
            $0$ & $1$ & $1$ & $0.10$ \\
            $1$ & $0$ & $0$ & $0.14$ \\
            $1$ & $0$ & $1$ & $0.18$ \\
            $1$ & $1$ & $0$ & $0.19$ \\
            $1$ & $1$ & $1$ & $0.11$ \\
        \bottomrule
    \end{tabular}
    }
    \end{minipage}
    \begin{minipage}{0.75\linewidth}
        \begin{questions}
            \question What is the marginal distribution of $A$?
            \answer
            \question What is the conditional distribution of $A$ given $B=1$ and $C=1$?
            \answer
        \end{questions}
    \end{minipage}
    

    \question 1\% of women at age forty who participate in routine screening have breast cancer. 80\% of women with breast cancer will get positive mammographies. 9.6\% of women without breast cancer will also get positive mammographies. A woman in this age group had a positive mammography in a routine screening. What is the probability that she actually has breast cancer?

    \answer

    % \question Label the following statements with a \cmark{} if they are proper elements or consequences of standard probability theory or with an \xmark{} if they are incorrect.
    %     \begin{questions}
    %         \question The probability of an event A must be greater than or equal to 0
    %         \question If $S$ is a sample space, then $P(S)=1$
    %         \question The Law of Averages: if you end up with 5 tails in a row on a series of coin flips, then the probability of getting heads on the next flip is greater than $\frac{1}{2}$.
    %         \question The Law of Large Numbers: the relative frequency of an event approaches the probability of that event occurring in the limit of an infinite number of sample experiments.
    %         \question The Law of Uniformity: all events on a sample space have the same probability of occurring
    %         \question If $A$ and $B$ are events, then $P(A \cup B) = P(A) + P(B) - P(A \cap B)$
    %     \end{questions}  

    % \question Briefly explain the Central Limit Theorem in your own words.

    % \answer

\end{questions}

\end{document}
