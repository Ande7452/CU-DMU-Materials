\documentclass{article}
\usepackage{fullpage,amsmath,amsthm,graphicx,enumitem,amssymb}
\usepackage[hidelinks]{hyperref}
\theoremstyle{definition}
\newtheorem{thm}{Theorem}
\newtheorem{question}[thm]{Question}
\newenvironment{solution}{\noindent\textit{Solution:}}{}

\newcommand{\reals}{\mathbb{R}}

\title{ASEN 5519-003 Decision Making under Uncertainty\\
       Homework 3: Online MDP Methods}

\begin{document}

\maketitle

\section{Conceptual Questions}

% Draw trees
% Read Sparse Sampling Paper
% Continuous Q value
% gamma down to epsilon
% \begin{question}
%     Consider an MDP with a discount factor of 0.9 and rewards between 0 and 1. Suppose that we wish to construct a complete MDP planning tree that estimates the value of the 
% \end{question}

% \begin{question}
%     % Consider an MDP with three states, $\mathcal{S} = \{s^1, s^2, s^3\}$ and two actions, $\mathcal{A} = \{a^1, a^2\}$. Draw the following three trees to a depth of $d=2$, with circles representing state nodes and squares representing action nodes:
%     (20 pts) Consider an MDP with three states, $|\mathcal{S}| = 3$ and two actions, $|\mathcal{A}| = 2$. Draw the following three trees to a depth of $d=2$, with circles representing state nodes and squares representing action nodes:
%     \begin{enumerate}[label=(\alph*)]
%         \item The complete state-action tree produced with forward search.
%         \item A sparse sampling tree with $n=2$.
%         \item A partial tree after 4 iterations of Monte Carlo tree search (according to Algorithm 4.9).
%     \end{enumerate}
% 
% \end{question}

% \begin{question}
%     (10 pts) Consider an MDP with $\mathcal{S} = \{1, 2\}$, $\mathcal{A} = \{1, 2\}$, $\gamma = 0.9$. If $Q*(1, 1) = $, $V*(2) = $
% \end{question}

% \begin{question}
%     (10 pts) Consider an MDP with $-10 \leq R(s, a) \leq 10$ and $\gamma = 0.9$. You are trying to evaluate two policies, $\pi_a$ and $\pi_b$, using Monte Carlo simulations.
%     \begin{enumerate}[label=(\alph*)]
%         \item If you want to use finite-horizon simulations of $T$ steps to generate a sample of the infinite-horizon return with an error of no more than $\epsilon = 0.01$, how many steps do you need to simulate?
%         \item Suppose that you run 100 $T$-step simulations each for $\pi_a$ and $\pi_b$. The 100 return samples for $\pi_a$ have a sample mean of 1.3 and sample standard deviation of 1.4. The 100 returns for $\pi_b$ have a sample mean of 1.1 and a sample standard deviation of 0.7. Approximately how many additional simulations should you run to be reasonably confident that $\pi_a$ is better than $\pi_b$ -- $100$, $10000$, $1\times10^6$, or $1\times10^9$?
%     \end{enumerate}
% \end{question}

\begin{question}
    (30 pts) In the proof for Lemma 5 of the Sparse Sampling paper by Kearns, Mansour, and Ng,\footnote{\url{https://www.cis.upenn.edu/~mkearns/papers/sparsesampling-journal.pdf}; Note: you do not need to read the paper to complete the problem.}, the authors claim that if a policy $\pi$ satisfies $|Q^*(s, \pi^*(s)) - Q^*(s, \pi(s))| \leq \beta$ for all $s \in \mathcal{S}$, then it immediately follows that $|R(s, \pi^*(s)) - R(s, \pi(s))| \leq \beta$. This statement is mistaken. Provide an MDP and a policy that present a counterexample to this claim and demonstrate that the statement does not hold.
\end{question}

\section{Exercises}

\begin{question}\label{q:mcts}
    (30 pts) Monte Carlo Tree Search
    
    Write code that performs 7 iterations of Monte Carlo Tree Search for an MDP created with \texttt{HW3.DenseGridWorld()} starting at state $(19, 19)$. You will need to produce three dictionaries:
    \begin{itemize}[noitemsep]
        \item \texttt{Q} maps $(s, a)$ tuples to Q value estimates.
        \item \texttt{N} maps $(s, a)$ tuples to N, the number of times the node has been tried.
        \item \texttt{t} maps $(s, a, s')$ tuples to the number of times that transition was generated during construction of the tree.
    \end{itemize}
    Then visualize the resulting tree with \texttt{HW3.visualize\_tree(Q, N, t, SA[19, 19])}\footnote{\texttt{SA} is from the \texttt{StaticArrays.jl} package.}. \textbf{Submit an image of the tree, the code used to generate it, and a few sentences describing the tree after 7 iterations} (e.g. which actions have the highest Q values? Does this make sense?).
    
    You will need to use the following functions from POMDPs.jl for the problem:
    \begin{itemize}[noitemsep]
        \item \texttt{actions(m)}
        \item \texttt{@gen(:sp, :r)(m, s, a)}
        \item \texttt{isterminal(m, s)}
        \item \texttt{discount(m)}
        \item \texttt{statetype(m)}
        \item \texttt{actiontype(m)}
    \end{itemize}

    You may also wish to use \texttt{POMDPs.simulate} and \texttt{POMDPSimulators.RolloutSimulator} for the rollouts.

    \texttt{HW3.DenseGridWorld()} randomly generates a 100x100 grid world problem. There is a reward of +100 every 20 cells, i.e. at [20,20], [20,40], [40,20], etc. Once the agent reaches one of these reward cells, the problem terminates. All cells also have a randomly generated cost.
\end{question}

\section{Challenge Problem}

\begin{question}
    (20 pts code and description, 20 pts score) Fast Online Planning
    
Create a function \texttt{select\_action(m,s)} that takes in a \texttt{DenseGridWorld}, \texttt{m}, and a state \texttt{s}, and returns a near-optimal action within 50ms. You may wish to base this code on the MCTS code that you wrote for Question~\ref{q:mcts}. Evaluate this function with \texttt{HW3.evaluate} and \textbf{submit the resulting \texttt{json} file along with the code and a one paragraph to one page description of your approach}, including tuning parameters that worked well, the rollout policy, etc. A score of 50 will receive full credit. There are no restrictions on this problem - you may wish to use a different algorithm, multithreading, etc. Starter code will be linked to on Canvas that will give suggestions for timing and other details.
\end{question}

\end{document}
